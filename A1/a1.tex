
\documentclass{article}
\usepackage[utf8]{inputenc}
\usepackage{amsmath}
\usepackage{algorithm}
\usepackage{algpseudocode}

\usepackage[margin=2.2cm]{geometry}

\title{Theory Assignment-1: ADA Winter-2024}
\author{Vikranth Udandarao (2022570) \and Ansh Varshney (2022083)}

\date{}
\begin{document}

\maketitle

\section{Preprocessing}
In this problem, no specific preprocessing steps are required as the input arrays A, B, and C are assumed to be sorted in ascending order. We have assumed/have been given the following: 1. A,B,C are sorted arrays.\2. Indices for all arrays start from 0 and end at n-1.\3.

\section{Algorithm Description}
The algorithm uses a binary search approach to find the k-th smallest element in the union of three sorted arrays A, B, and C. It efficiently narrows down the search space based on the comparison of mid-values in the arrays.

\section{Recurrence Relation}
The algorithm doesn't involve recursion, so there is no recurrence relation.

\section{Complexity Analysis}
The time complexity of the algorithm is O(log n), where n is the size of the input arrays A, B, and C. This is achieved through binary search operations on each array and subsequent convergence of the search space.

The space complexity is O(1) as the algorithm uses a constant amount of additional space for variables.

\section{Pseudocode}
\begin{algorithm}
\caption{Kth Smallest Element}
\begin{algorithmic}[1]
    \Function{KthSmallestElement}{$A, B, C, k$}
        \State $low \gets \min(A[0],B[0],C[0])$
        \State $high \gets \max(A[n-1], B[n-1], C[n-1])$
        
        \While{$low < high$}
            \State $mid \gets low + (high - low) / 2$
            \State $a\_count \gets$ \Call{BinarySearchCount}{$A, mid$}
            \State $b\_count \gets$ \Call{BinarySearchCount}{$B, mid$}
            \State $c\_count \gets$ \Call{BinarySearchCount}{$C, mid$}
            \State $total\_count \gets a\_count + b\_count + c\_count$
            
            \If{$total\_count < k$}
                \State $low \gets mid + 1$
            \Else
                \State $high \gets mid$
            \EndIf
        \EndWhile
        
        \State \Return $low$
    \EndFunction
\end{algorithmic}
\end{algorithm}

\section{Proof of Correctness}
The correctness of the algorithm lies in the binary search approach, which efficiently narrows down the search space until the k-th smallest element is found. The algorithm maintains correctness by ensuring that the counts of elements less than or equal to the mid-value in each array are correctly calculated.

\end{document}